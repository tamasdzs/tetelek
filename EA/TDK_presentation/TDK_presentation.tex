
\documentclass{beamer}
\usetheme{Darmstadt}
\usefonttheme[onlylarge]{structurebold}
\usepackage[T1]{fontenc}
\usepackage{t1enc}	
\usepackage{indentfirst}
\usepackage[utf8]{inputenc}
\usepackage[normalem]{ulem}
\usepackage[makeroom]{cancel}

\usepackage{graphicx}
\usepackage{graphics}
\usepackage{amssymb,amsmath}
\usepackage{animate}
\usepackage[super]{nth}
\usepackage[tight,footnotesize]{subfigure}
\usepackage{rotating}
\usepackage{bigstrut}

%EZT A CSOMAGOT MÉG ADD HOZZÁ!
\usepackage{multirow}

%\usepackage{subfig}
%\usepackage{wrapfig}
\linespread{1.5}


\newtheorem{tetel}{Tétel}
\DeclareMathOperator{\re}{Re}
\DeclareMathOperator{\im}{Im}
\DeclareMathOperator{\Span}{span}
\DeclareMathOperator{\thh}{th}
\DeclareMathOperator{\Hz}{Hz}
\DeclareMathOperator{\PLI}{PLI}
\DeclareMathOperator{\sgn}{sgn}
\DeclareMathOperator{\cur}{cur}
\DeclareMathOperator{\infx}{inf}
\DeclareMathOperator{\err}{err}
\DeclareMathOperator{\ido}{time}
\DeclareMathOperator{\grad}{grad}
\DeclareMathOperator{\intenzitas}{intenzitás}
\newcommand{\tss}[1]{\textsuperscript{#1}}
\newcommand{\IN}{\mathbb{N}}
\newcommand{\IZ}{\mathbb{Z}}
\newcommand{\IR}{\mathbb{R}}
\newcommand{\IC}{\mathbb{C}}
\newcommand{\ID}{\mathbb{D}}
\newcommand{\IT}{\mathbb{T}}
\newcommand{\IL}{\mathbb{L}}
\newcommand{\IH}{\mathbb{H}}
\newcommand{\IP}{\mathbb{P}}
\newcommand{\dt}{\ \mathrm{d}t}
\newcommand{\set}[2]{\left\{\, #1 \,:\, #2 \,\right\}}
\newcommand{\sset}[1]{\left\{\, #1 \,\right\}}
\newcommand{\conj}[1]{\overline{#1}}
\DeclareMathOperator{\arth}{arth}
\DeclareMathOperator{\tanhyp}{th}
\DeclareMathOperator{\dbest}{dbest}
\DeclareMathOperator{\PRD}{PRD}
\DeclareMathOperator{\CR}{CR}
\DeclareMathOperator{\QS}{QS}


\title{Biológiai jelek Hilbert-térbeli approximációja}
\author{Dózsa Tamás}
\institute{Programtervező Informatikus BSc \\ Eötvös Loránd  Tudományegyetem, Informatikai Kar\vspace{3mm}}
\date{\vspace{2mm}\\
\vspace{3mm}
\scriptsize
Budapest,\\ 2017. júlus 3.
%\textcolor{blue}{The Project is supported by the European Union and co-financed by the European Social
%Fund (grant agreement no.\ T\'AMOP 4.2.2./B-10/1-2010-0030).}
}
\usetheme{Darmstadt} 
%\setbeamerfont{caption}{size=1pt}
\begin{document}

\begin{frame}
\titlepage
\end{frame}

\begin{frame}{Tartalomjegyzék}
\tableofcontents%[pausesection] % a szakaszok egymás után jelennek meg
\end{frame}


\section{Bevezetés}
\begin{frame}
\only<1>{
	\begin{block}{Motiváció}
	\begin{itemize}
		\item Az EKG elemzése nagy jelentőséggel bír az orvostudományban.
		\item Hosszú mérések, vagy több elvezetés esetén költséges a tárolás.
		\item A jel zajjal terhelt, ami nehezíti a pontos diagnózis felállítását.
		\item Hullámszegmensek meghatározása fontos probléma.
	\end{itemize}
\end{block}
	\begin{block}{Alkalmazások}
	\begin{itemize}
		\item \textbf{Tömörítés}
		\item Zajszűrés
		\item QRS detektálás
		\item Szívütések osztályozása
	\end{itemize}
\end{block}
}
\only<2>{
\begin{center}
   \includegraphics[scale=0.35]{./figures/ecg_wiki.png}
 		\label{fig:ekg}
\end{center}
}
\end{frame}

\begin{frame}{Approximáció Hilbert-terekben}
\only<1>{
\begin{block}{Definíciók}
	\begin{itemize}
		\item Legyen $(\mathcal{H},\left\langle .,.\right\rangle)$ Hilbert-tér, ahol
			\begin{equation*}
				\left\langle f,g\right\rangle := \int_{-\infty}^{\infty} f(t) g(t) \dt
			\end{equation*}
		\item $\set{\Phi_k}{0\leq k \leq n}$ ortonormált rendszer (ONR)
		\item Ortogonális projekció:
			\begin{equation*}
				S_n f:=\sum_{k=0}^{n} \left\langle f,\Phi_k\right\rangle \Phi_k
			\end{equation*}
	\end{itemize}
\end{block}
}
\only<2>{
	\begin{block}{Definíciók}
	\begin{itemize}
		\item Az approximáció hibája:
			\begin{equation*}
			\|f-S_nf\|^2=\|f\|^2-\sum_{k=0}^n|\langle f,\Phi_k\rangle|^2 \,.
			\end{equation*}
	\end{itemize}
	\end{block}
	\begin{block}{Feladat}
		\begin{itemize}
		  \item EKG esetén másodpercenként több száz adat keletkezik.
			\item A dolgozat célja a jel \textbf{tömörítése.}
			\item Ezért véges dimenziós altereket használunk (Euklideszi tér).
		\end{itemize}
	\end{block}
}
\end{frame}

\section{Előzmények}

\begin{frame}{Hermite-polinomok}
\linespread{1.0}
\begin{block}{Definíció}
	\begin{itemize}
		\item Rekurzió:
			\begin{equation*}
				\begin{split}
					&H_n(x)=2xH_{n-1}(x)-2(n-1)H_{n-2}(x)\quad (n\ge 2)\\
					&H_{-1}(x)=0,\  H_0(x)=1,\ \ H_1(x)=2x
				\end{split}
			\end{equation*}
		\item Ortogonalitás:
			\begin{gather*}
				\begin{split}
					\langle H_n,H_m\rangle_{\rho}:&=\int_{-\infty}^\infty H_n(t) H_m(t)\rho(t) \dt\\
					\langle H_n,H_m\rangle_{\rho} &= 0 \quad (n \neq m)
				\end{split}
			\end{gather*}
		\item Súlyfüggvény:
			\begin{equation*}
				\rho(t) := e^{-t^2}
			\end{equation*}
	\end{itemize}
\end{block}
\end{frame}


\begin{frame}{Hermite-függvények}
\only<1>{
\begin{block}{Definíció}
		\begin{equation*}
			\Phi_n(x):=H_n(x)e^{-x^2/2}/\sqrt{\pi^{1/2}2^n n!} \quad (n\in\Bbb N, x\in\Bbb R)
		\end{equation*}
\end{block}
\begin{figure}
  \centering
\subfigure[$\Phi_{0}(x)$]{\includegraphics[scale=0.3,trim=150 280 150 280,clip]{./figures/phi0.pdf}} 
\subfigure[$\Phi_{1}(x)$]{\includegraphics[scale=0.3,trim=150 280 150 280,clip]{./figures/phi1.pdf}}
\subfigure[$\Phi_{2}(x)$]{\includegraphics[scale=0.3,trim=150 280 150 280,clip]{./figures/phi2.pdf}} 
\caption{A Hermite-f\"uggv\'enyrendszer első három tagja.}
\label{fig:phi0-3}
\end{figure}

}
\only<2>{
	\begin{block}{Tulajdonságok}
	\begin{itemize}
		\item ONR
		\item Gyorsan tartanak 0-hoz:
			\begin{equation*}
				|\Phi_n(x)|\le M_n e^{-x^2/4}\le M_n \quad (x\in\Bbb R, n\in\Bbb N)
			\end{equation*}
		\item	Stabil, másodrendű rekurzió:
			\begin{equation*}
				\begin{split}
					&\Phi_n(x)=\sqrt{\frac 2 n}x\Phi_{n-1}-\sqrt{\frac {n-1} n}\Phi_{n-2}(x)\quad (n\ge 2)\\ 
					&\Phi_0(x)=e^{-x^2/2}/\pi^{1/4},\ \Phi_1(x)= \sqrt 2 x e^{-x^2/2}/\pi^{1/4}\quad (x\in\Bbb R)
				\end{split}
			\end{equation*}	
	\end{itemize}
	\end{block}
}
\only<3>{
\linespread{1.0}
	\begin{block}{Tulajdonságok}
	\begin{itemize}
		\item $\Phi_n$ deriváltja:
			\begin{equation*}
				\Phi'_n(x)=\sqrt{2n}\Phi_{n-1}(x)-x\Phi_n(x) \quad (n\ge 0, x\in\Bbb R)
			\end{equation*}
		\item	Alakjuk korrelál az EKG-vel.
		\item Korábbi cikkek is használták \footnotemark[1] \footnotemark[2].
	\end{itemize}
	\end{block}
\footnotetext[1]{R. Jane, S. Olmos, P. Laguna, and P. Caminal, \textit{Adaptive Hermite models for ECG data compression: Performance and evaluation with automatic wave detection}, in Proc. of the Int. Conf. on Comp. in Card., 1993, pp. 389--392. \vspace{3mm}}	
\footnotetext[2]{A. Sandryhaila, S. Saba, M. Püschel, and J. Kovacevic, \textit{Efficient compression
of QRS complexes using Hermite expansion}, IEEE Trans. on Sig. Proc., vol. 60, no. 2, pp. 947–955, 2012. \vspace{3mm}}	
}
\end{frame}

\section{Kidolgozott módszer}

\begin{frame}{Rendszerek affin transzformáltja}
\only<1>{
	\begin{block}{Ötlet}
	\begin{itemize}
		\item Transzlációs, és dilatációs paraméterek bevezetése:
			\begin{equation*}
				\Phi_k^{a,\lambda}(x):=\Phi_k(\lambda x+a) \quad  (x,a\in\Bbb R, \lambda>0)\,.
			\end{equation*}
		\item	$\set{\sqrt{\lambda}\Phi_k^{a,\lambda}}{0\leq k \leq n}$ is ONR.
		\item Legjobb közelítés:
			\begin{equation*}
				S_n^{a,\lambda}f:=\sum_{k=0}^n\langle f,\Phi_k^{a,\lambda}\rangle\Phi_k^{a,\lambda}\quad
			(n\in\Bbb N, a\in\Bbb R,\lambda>0)\,.
			\end{equation*}
			\item Implementációs szempontok miatt $f^{a,\lambda}$-t használjuk.
	\end{itemize}
	\end{block}
}
\only<2>{
\linespread{1.0}
	\vspace{-2mm}
	\begin{block}{Célfüggvény definíciója}
	\begin{itemize}
		\item A Bessel egyenlőtlenség alapján a hibaformula:
			\begin{equation*}
				D^2_n(a,\lambda):=\|f\|^2-\sum_{k=0}^n|\langle f,\Phi_k^{a,\lambda}\rangle|^2\,.
			\end{equation*}
		\item Használható továbbá a
			\begin{equation*}
				F_n(a,\lambda):=\sum_{k=0}^n|\langle f,\Phi_k^{a,\lambda}\rangle|^2\,.
			\end{equation*}
	\end{itemize}
	\end{block}
	\begin{block}{Problémák}
	\begin{itemize}
		 \item Létezik-e a hibafüggvényeknek minimum, maximum helye?
		 \item Nem triviális: az értelmezési tartomány nem kompakt halmaz.
	\end{itemize}
	\end{block}	
}
\end{frame}

\begin{frame}{Optimalizációs módszerek}
\only<1>{
	\begin{block}{Cél}
		A legjobb transzlációs, dilatációs paraméterek megtalálása.
	\end{block}
	\begin{block}{Módszerek}
		\begin{itemize}
			\item Determinisztikus:
				 \begin{itemize}
						\item Gradiens módszer (parciális deriváltak szükségesek)
						\item \textbf{Nelder--Mead (NM) szimplex alapú módszer} 
				 \end{itemize}
			\item Nem determinisztikus:
				 \begin{itemize}
						\item Particle Swarm Optimization (PSO), raj alapú optimalizáció
				 \end{itemize}
		\end{itemize}
	\end{block}	
}
\end{frame}


\begin{frame}{Matching Pursuit algoritmus}
\only<1>{
	\begin{block}{Cél}
		\begin{itemize}
			\item Egy szívütés – több függvényrendszer.
			\item Mohó algoritmus, minden körben a hibafüggvény minimumát keresi (a reziduum jelre nézve).
		\end{itemize}
	\end{block}	
	\begin{block}{Előnyök}
		\begin{itemize}
			\item Alkalmas a szegmensek detektálására.
			\item Adaptív reprezentáció.
			\item Pontosabb approximáció kevesebb együtthatóval.
			\item A dolgozatban használt együtthatók száma: 7, 6, 2 .
		\end{itemize}
	\end{block}		
}
\only<2>{
	\begin{figure}
  \animategraphics[autoplay,loop,scale=0.45,trim=100 210 100 260]{4}{./Animation/hermiteECG_}{0}{50}\vspace{-2mm}
		\caption{QRS paramétereinek optimalizációja.}
	\end{figure}	
}
\only<3>{
	\begin{figure}
  \animategraphics[autoplay,loop,scale=0.45,trim=100 210 100 260]{4}{./Animation/hermiteECG_}{107}{157}\vspace{-2mm}
		\caption{T paramétereinek optimalizációja.}
	\end{figure}	
}
\only<4>{
	\begin{figure}
  \animategraphics[autoplay,loop,scale=0.45,trim=100 210 100 260]{4}{./Animation/hermiteECG_}{246}{296}\vspace{-2mm}
		\caption{P paramétereinek optimalizációja.}
	\end{figure}	
}
\only<5>{
\begin{figure}[htb!]
  \centering
\subfigure[A QRS approxim\'aci\'oja.]{\includegraphics[scale=0.4,trim=120 280 100 280,clip]{./figures/abra_lepes0.pdf}} 
\subfigure[A T hull\'am approxim\'aci\'oja.]{\includegraphics[scale=0.4,trim=120 280 100 280,clip]{./figures/abra_lepes1.pdf}}
\caption{Az MP algoritmus lépései (1-2).}
\end{figure}
}
\only<6>{
\begin{figure}[htb!]
  \centering
\subfigure[A P hull\'am approxim\'aci\'oja.]{\includegraphics[scale=0.4,trim=120 280 100 280,clip]{./figures/abra_lepes2.pdf}} 
\subfigure[Szeparált szívütés.]{\includegraphics[scale=0.4,trim=120 280 100 280,clip]{./figures/abra_lepes3.pdf}}
\caption{Az MP algoritmus lépései (3-4).}
\end{figure}
}
\end{frame}

\section{Implementáció}
\label{tab:implemenation}

\begin{frame}{A tömörítés implementációja}
\linespread{0.8}
\small
\vspace{-2mm}
\begin{block}{Modulok}
	\begin{itemize}
		\item Objektum orientált felépítés.
		\item Jól definiált interfészek.
		\item Elkülönített szolgáltatások.
		\item Újrafelhasználhatóság, könnyen bővíthetőség.
	\end{itemize}
	\end{block}		
	
\end{frame}

\begin{frame}{A tömörítés implementációja}
\linespread{0.8}
\small
\vspace{-2mm}
\begin{block}{Alkalmazott technikák, érdekességek}
	\begin{itemize}
		\item Összesen 9 c++ nyelven megírt modul.
		\item Algoritmusok: Nelder-Mead, MP, Hermite függvény gyökeinek megtalálása.
		\item c++ 11 adottságainak felhasználása: lambda függvények. 
	\end{itemize}
	\end{block}		
	
\begin{block}{A hatékonyság szem előtt tartása}
	\begin{itemize}
		\item Dinamikus memória használata (Hermite rendszerek).
		\item c++ sajátottságainak kihasználása, adatábrázolás során.
		\item A Standard Library szolgálatatásainak igénybevétele. 
	\end{itemize}
\end{block}		
\end{frame}

\begin{frame}{A felhasználói felület}
\linespread{0.8}
\small
\vspace{-2mm}
\begin{block}{Alkalmazott technikák}
	\begin{itemize}
		\item Feladatspecifikus nyelvválasztás. Felhasznált programnyelvek: php, javascript, bash.
		\item Külső technológiák integrálása: GoogleCharts, jQuerry.
	\end{itemize}
	\end{block}		
	
\begin{block}{Kényelem orientált felépítés}
	\begin{itemize}
		\item Átlátható horizontális menüsor.
		\item Ismertető felület.
		\item Esztétikus megjelenés. Tervező grafikus által szerkesztett külalak. 
	\end{itemize}
\end{block}		

\end{frame}



\section{Tesztek, eredmények}
\label{tab:results}

\begin{frame}{A programcsomag tesztelése}
\linespread{0.8}
\small
\vspace{-2mm}
\begin{block}{Modul tesztek}
	\begin{itemize}
		\item Minden modulhoz külön modulteszt.
		\item Eredmények, modulteszt kódja: ./src/mt/[modul neve]
	\end{itemize}
	\end{block}		
\end{frame}

\begin{frame}{Tesztelés}
\only<1>{
\begin{block}{Értékelés}
	\begin{itemize}
		\item Percentage root mean square difference (PRD):
			\begin{equation*}
					\PRD=\frac{\left\|S_{n}^{a,\lambda}f-f\right\|_2}{\left\|f-\conj{f}\right\|_2}\times 100\,.  		
			\end{equation*}		
		\item Compression ration (CR):
			\begin{equation*}
				\CR=\frac{\text{eredeti EKG mérete}}{\text{tömörített EKG mérete}}\times 100\,.
			\end{equation*}		
		\item Quality score (QS):
			\begin{equation*}
				\QS=\frac{CR}{PRD}\,.
			\end{equation*}		
	\end{itemize}
\end{block}
}
\only<2>{
	\begin{block}{Technikai adatok}
		\begin{itemize}
			\item PhysioNet EKG adatbázisból $6$ egyenként fél órás rekordon.
			\item $12830$ szívütés, rekordonként 1400-1600 másodperces futásidő.
		\end{itemize}
	\end{block}
\begin{table}[H]
\scriptsize
\centering
	\scalebox{0.7}{
\begin{tabular}{|c||cccc|cccc|cccc|}
\multicolumn{1}{c}{} & \multicolumn{4}{c}{\textbf{Hiba (PRD \%)}} & \multicolumn{4}{c}{\textbf{A t\"om\"or\'\i t\'es ar\'anya (CR $1:X$)}} & \multicolumn{4}{c}{\textbf{Quality Score ($\CR:\PRD$)}}\bigstrut\\\hline
\textbf{Rec.} & \textbf{Eredeti} & \textbf{NM} & \textbf{PSO} & \textbf{JPEG2} & \textbf{Eredeti} & \textbf{NM} & \textbf{PSO} & \textbf{JPEG2} & \textbf{Eredeti} & \textbf{NM} & \textbf{PSO} & \textbf{JPEG2} \bigstrut[t]\\
101 & 11.20 & 11.10 & 11.13 &11.07& 29.71 & 27.22 & 27.22 & 18.94 &\textbf{2.65}&2.45&2.47& 1.71\\
117 & 13.20 & 11.81 & 17.66 &11.66& 36.07 & 33.06 & 33.06 & 23.66 &2.73&\textbf{2.79}&1.87& 2.02\\
118 & 19.83 & 17.79 & 16.65 &17.72& 24.34 & 22.30 & 22.30 & 32.91 &1.22&1.25&1.33& \textbf{1.85}\\
\textcolor{red}{119} & \textcolor{red}{14.27} & \textcolor{red}{8.76} & \textcolor{red}{10.20}  &\textcolor{red}{8.80} & 27.89 & 25.55 & 25.55 & 23.85 &1.95&\textbf{2.91}&2.51& 2.71\\
201 & 13.51 & 12.17 & 12.17 &12.14& 28.21 & 25.35 & 25.35 & 13.15 &\textbf{2.08}&\textbf{2.08}&\textbf{2.08}& 1.08\\
213 & 19.92 & 18.28 & 17.60 &18.29& 17.08 & 15.64 & 15.64 & 35.23 &0.85&0.85&0.88& \textbf{1.92}\\
%214 & 7.61 & 7.90 & 11.40   &7.91 & 24.53 & 22.04 & 22.47 & 19.32 &&&&\\
\hline
\textbf{\'Atlag} & 14.22 & 12.55 & 13.83 &12.51& 26.83 & 24.45 & 24.51 & 23.86 &1.91&\textbf{2.06}&1.85& 1.88\\\hline
\end{tabular}
}
\caption{A tömörítés \"osszehasonl\'\i t\'asa k\"ul\"onb\"oz\H o m\'odszerek eset\'en.}
\label{tab:results}
\end{table}
}
\only<3>{
\begin{figure}[htb!]
  \centering
\subfigure[Eredeti módszer.]{\includegraphics[scale=0.4,trim=120 280 100 280,clip]{./figures/eredeti_abra1.pdf}}\hspace{1mm}
\subfigure[Saját módszer.]{\includegraphics[scale=0.4,trim=120 280 100 280,clip]{./figures/sajat_abra1.pdf}}
\caption{Asszimetrikus EKG jel közelítése (119-es rekord).}
\end{figure}
}
\end{frame}

\begin{frame}{Összefoglalás}
\linespread{0.8}
\small
\vspace{-2mm}
\begin{block}{Eredmények}
	\begin{itemize}
		\item Új ONR szerkesztése transzlációval és dilatációval.
		\item Optimum létezésének belátása.
		\item Alkalmazás EKG jelek tömörítésére, szegmentálására.
		\item Hatékony implementáció elkészítése, webes felhasználói felülettel.
		\item Módszer publikálása nemzetközi folyóiratban \cite{sajat}.
		\item Eredmények bemutatása nemzetközi konferencián.
		\item Eredmények bemutatása Tudományos Diákköri Konferencián.
	\end{itemize}
	\end{block}				
	\begin{block}{Konklúzó}
	\begin{itemize}
		\item A módszer hatékonyan alkalmazható EKG jelek tömörítésére.
		\item Pontosabb közelítés érhető el asszimetrikus jelek esetén.
		\item Közel valósidejű feldolgozás lehetséges.	
	\end{itemize}
	\end{block}	
\end{frame}

\begin{frame}{Irodalomjegyzék}
\begin{thebibliography}{9}
\linespread{1.0}

\bibitem{Szoke} B.~Szokefalvi-Nagy, \lq\lq Valós függvények és függvénysorok,\rq\rq~in~\emph{Polygon Könyvtár}, Szeged, HU, 2002.

\bibitem{Szego} G.~Szeg\H{o}, \lq\lq Orthogonal polynomials,\rq\rq~\emph{AMS Colloquium Publications}, New York, USA, 3rd edition, 1967.

\bibitem{Gautschi} W.~Gautschi, \lq\lq Orthogonal Polynomials, Computation and Approximation,\rq\rq~in~\emph{Numerical Mathematics and Scientific Computation}, Oxford University Press, Oxford,
UK, 2004.

\bibitem{Mallat} S.~G.~Mallat, Z.~Zhang, \lq\lq Matching pursuit in time-frequency dictionary,\rq\rq~in~\emph{IEEE Transactions on Signal Processing}, vol. 41, no. 12, pp. 3397–3415, 1993.

\bibitem{sajat} 
T. Dozsa, P. Kovacs,
„ECG Signal Compression Using Adaptive Hermite Functions,”
\textit{ Advances in Intelligent Systems and Computing},
AISC, volume 399

%\bibitem{PhysioNet} A.~L.~Goldberger, L.~A.~N.~Amaral, L.~Glass, J.~M.~Hausdorff, P.~Ch.~Ivanov,
%R.~G.~Mark, J.~E.~Mietus, G.~B.~Moody, C.~K.~Peng, and H.~E.~Stanley, \lq\lq PhysioBank,
%PhysioToolkit, and PhysioNet: Components of a new research resource
%for complex physiologic signals,\rq\rq~in~\emph{Circulation}, vol. 101, no. 23, pp. 215–220, 2000.

\end{thebibliography}
\end{frame}

\begin{frame}{}
\end{frame}


\begin{frame}{Kitekintés}
\begin{figure}[htb!]
  \centering
\subfigure[Rossz eset.]{\includegraphics[scale=0.38,trim=100 280 100 280,clip]{./figures/abra762_bad.pdf}}
\subfigure[Jó eset.]{\includegraphics[scale=0.38,trim=100 280 100 280,clip]{./figures/abra762_good.pdf}}
\caption{A mohó stratégia hátránya.}
\end{figure}
\end{frame}

\begin{frame}{Nelder--Mead szimplex módszer}
\only<1>{
	\begin{block}{Cél}
		A $D^2_n(a,\lambda)$ hibafüggvény minimumhelyének meghatározása.
				%\begin{equation*}
					%D^2_n(a,\lambda):=\|f\|^2-\sum_{k=0}^n|\langle f,\Phi_k^{a,\lambda}\rangle|^2\,.
				%\end{equation*}
	\end{block}
	\begin{block}{Algoritmus}
		\begin{itemize}
			\item Kizárólag az $ f(x_3)\le f(x_2)\le f(x_1)$ értékekre támaszkodik.
			\item Keressük az $x'=x_1+\alpha (x-x_1)\ \ (\alpha\in\Bbb R)$ pontot, melyre
			\begin{equation*}
				f(x')\le f(x_2)\le f(x_1)\,,
			\end{equation*}
			ahol $x=(x_2+x_3)/2$ a két kisebb érték között húzott szakasz felezőpontja.
		\end{itemize}
	\end{block}
	}
\only<2>{
\small
\linespread{0.8}
		\begin{block}{Geometriai transzformációk}
		\begin{itemize}
			\item $\alpha=2$ esetén pont egy $x$-re vonatkozó középpontos tükrözés ($T_1$).
			\item $\alpha>2$ esetén a tükrözés és nyújtás $(T_2)$.
			\item $1<\alpha<2$ esetén tükrözés és zsugorítás $(T_3)$.
			\item $-1<\alpha<0$ esetén egyszerű zsugorítás ($T_4$). 
		\end{itemize}
	\end{block}
\vspace{-3mm}
\begin{figure}[htb!]
  \centering
\subfigure[T\"ukr\"oz\'es\ $(T_1: \alpha=2).$]{\includegraphics[scale=0.22,trim=10 0 10 53,clip]{./figures/NMT1.PNG}} \hspace{10mm}
\subfigure[T-Ny\'ujt\'as\ $(T_2: \alpha=2.5).$]{\includegraphics[scale=0.22,trim=10 0 10 53,clip]{./figures/NMT2.PNG}}
\end{figure}
}
\only<3>{
\small
\linespread{0.8}
\begin{figure}[htb!]
\subfigure[T-\"Osszeh\'uz\'as\  $(T_3:  \alpha=1.5).$]{\includegraphics[scale=0.22,trim=10 0 10 53,clip]{./figures/NMT3.PNG}} 
\subfigure[Összeh\'uz\'as\ $(T_4: -1<\alpha<0).$]{\includegraphics[scale=0.22,trim=10 0 10 185,clip]{./figures/NMT4.PNG}}
\subfigure[Kicsiny\'\i t\'es  $x_3$-b\'ol $(T_5).$]{\includegraphics[scale=0.22,trim=10 0 10 185,clip]{./figures/NMT5.PNG}}
\end{figure}
}
\end{frame}

\begin{frame}{Diagnosztikai torzulás jellemzése}
\only<1>{
\linespread{1.0}
	\begin{block}{Alternatívák}
	\begin{itemize}
		\item WDD (2000): orvosdiagnosztikában használt jellemzők közvetlen átvétele \footnotemark[1].
			\begin{itemize}
				\item Hátrány: a felhasznált feature-k detektálása nagyon nehéz. 
			\end{itemize}
		\item WWPRD (2006): wavelet együtthatók súlyozott PRD-je \footnotemark[2].
   		\begin{itemize}
				\item Előny: könnyen implementálható, csak QRS detektálás szükséges.
			\end{itemize}
	\end{itemize}
	\end{block}
\footnotetext[1]{Y. Zigel, A. Cohen, and A. Katz, \textit{The Weighted Diagnostic Distortion (WDD) Measure
for ECG Signal Compression}, IEEE Trans. on Biomed. Eng., vol. 47, no. 11, pp. 1422--1430, 2000. \vspace{3mm}}
	
\footnotetext[2]{A. S. Al-Fahoum, \textit{Quality Assessment of ECG Compression Techniques Using a Wavelet-Based Diagnostic Measure}, IEEE Trans. on Inf. Tech. in Biomed., vol. 10, no. 1, pp. 182--191, 2006. \vspace{3mm}}	
}

\only<2>{
%	\begin{block}{Diagnostic Features (WDD)}
%		$RR_{int}, QRS_{dur}, QT_{int}, QTp_{int}, P_{dur}, PR_{int}, QRS_{peaks}, QRS_{sign}, \Delta_{wave?},$
%		$T_{shape}, P_{shape}, ST_{shape}, QRS^+_{amp}, QRS^-_{amp}, P_{amp}, T_{amp}, ST_{elevation}, ST_{slope}$
%	\end{block}	
	\begin{figure}
		\begin{center}
			\includegraphics[scale=0.27]{./figures/ECGfeatures.png}
		\end{center}
		\caption{Egy szívütés diagnosztikai jellemzői (WDD).}
	\end{figure}	
}
\only<3>{
\begin{table}[t!]
\centering
\scriptsize
\begin{center}
	\begin{tabular}{|c|c|c|c|c|c|} \hline
		\multirow{2}{*}{\textbf{Mérték}} & \multicolumn{5}{|c|}{\textbf{Minősítés}} \\ \cline{2-6}
		\textbf{} & \textbf{Tökéletes} & \textbf{Nagyon jó}  & \textbf{Jó} & \textbf{Nem Rossz} & \textbf{Rossz} \\ \hline
		\textbf{PRD} & 0-4.33 & 4.33-7.8 & 7.8-11.59 & 11.59-22.5 & $22.5<$ \\ \hline
		\textbf{$\text{WWPRD}_h$} & 0-7.4 & 7.4-15.45 & 15.45-25.18 & 25.18-37.4 & $37.4<$\\ \hline
	\end{tabular}
\end{center}
\label{tab:quality_pred}
\caption{Tömörítés diagnosztikai minősége PRD és $\text{WWPRD}_h$ alapján.}
\end{table}
}
\end{frame}
\end{document}



