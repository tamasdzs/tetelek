\documentclass[margin=0px]{article}

\usepackage{listings}
\usepackage[utf8]{inputenc}
\usepackage{graphicx}
\usepackage{float}
\usepackage[a4paper, margin=1in]{geometry}
\usepackage{amsthm}
\usepackage{amssymb}

\newenvironment{tetel}[1]{\paragraph{#1 \\}}{}
% A dokument itt kezdődik

\title{Záróvizsga tételsor \\ \large 5. Valószínűségszámítási és statisztikai alapok}
\date{}
\author{Fekete Dóra}

\begin{document}

	\maketitle
	
	\begin{tetel}{Valószínűségszámítási és statisztikai alapok}
			Diszkrét és folytonos valószínűségi változók, nagy számok törvénye, centrális határeloszlás tétel. Statisztikai becslések, klasszikus statisztikai próbák.
	\end{tetel}
	
	\section{Kolmogorov-féle valószínűségi mező}
	
	$(\Omega, \mathcal{A}, P)$ hármas, ahol:
	\begin{itemize}
		\item [$\Omega$] nem üres halmaz, eseménytér, $\omega$ elemi esemény.
		\item [$\mathcal{A}$] $\Omega$ részhalmazainak egy rendszere, $\mathcal{A} \subset 2^{\Omega}, A \in \mathcal{A}$ események, $\mathcal{A}$ $\sigma$-algebra. \\
		$\sigma$-algebra:
		\begin{enumerate}
			\item $\Omega \in \mathcal{A}$
			\item $A \in \mathcal{A} \Rightarrow \overline{A} = \Omega \backslash A \in \mathcal{A}$
			\item $A_{1}, A_{2}... \in \mathcal{A} \Rightarrow \bigcup_{i=1}^{\infty}{A_{i}} \in \mathcal{A}$
		\end{enumerate}
		\item [$P$] $: \mathcal{A} \to [0,1]$ halmazfüggvény, valószínűség, amelyre $P(\Omega) = 1, P(A) \geq 0, \forall A \in \mathcal{A}$-ra, páronként kizáró ($A_{i} \cdot A_{j} = \emptyset, i \neq j$) $A_{1}, A_{2}... \in \mathcal{A}$ eseményekre $P(\bigcup_{i=1}^{\infty}{A_{i}}) = \sum_{i=1}^{\infty}{P(A_{i})}$.
	\end{itemize}
	
	\section{Diszkrét és folytonos valószínűségi változók}
	
	\begin{itemize}
		\item \textit{Valószínűségi változó}: $\xi : \Omega \to \mathbb{R}$ mérhető függvény, azaz amire $\{\omega : \xi(\omega) < x\} \in \mathcal{A}, \forall x \in \mathbb{R}$, ahol $\mathcal{A}$ az eseménytér ($\Omega$) részhalmazainak egy rendszere. ($\omega \in \Omega$ elemi esemény).
		\item \textit{Valószínűségi változó eloszlása/eloszlásfüggvénye}: $F_{\xi}(x) = P(\xi < x)$, $\forall x \in \mathbb{R}$. \\
		Tulajdonságai:
		\begin{enumerate}
			\item $0 \leq F_{\xi}(x) \leq 1$
			\item monoton növő
			\item balról folytonos
			\item $\lim\limits_{x \to -\infty}{F_{\xi}(x)} = 0$, $\lim\limits_{x \to \infty}{F_{\xi}(x)} = 1$
		\end{enumerate}
	\end{itemize}
	
	\subsection{Diszkrét valószínűségi változók}
	
	Értékkészlete legfeljebb megszámlálhatóan végtelen, azaz $\{x_1 ... x_n ... \}$ elemekből áll. Ekkor eloszlása: $p_k := P(\xi = x_k)$.
	
	\begin{tabular}{|p{2.5cm}|p{4cm}|p{4cm}|c|c|}
		\hline \textbf{Név} & \textbf{Értelmezés} & \textbf{Eloszlás} & \textbf{$EX$} & \textbf{$D^{2}X$} \\ 
		\hline indikátor \newline $Ind(p)$ & Egy $p$ valószínűségű esemény bekövetkezik-e vagy sem. & $P(X=1) = p$ \newline $P(X=0) = 1-p$ & $p$ & $p(1-p)$ \\ 
		\hline geometriai (Pascal) \newline $Geo(p)$ & Hányadikra következik be először egy $p$ valószínűségű esemény. & $P(X=k) = p(1-p)^{k-1}$ \newline $k=1,2...$ & $\frac{1}{p}$ & $\frac{1-p}{p^2}$\\ 
		\hline hipergeometriai \newline $Hipgeo(N,M,n)$ & Visszatevés nélküli mintavétel. & $P(X=k) = \frac{{M \choose k}{N-M \choose n-k}}{{N \choose n}}$ \newline $k=0,1,...,n$ & $n \frac{M}{N}$ & $n \frac{M}{N}(1 - \frac{M}{N})(1 - \frac{n-1}{N-1})$ \\ 
		\hline binomiális \newline $Bin(n,p)$ & Visszatevéses mintavétel. & $P(X=k) = {n \choose k}p^{k}(1-p)^{n-k}$ \newline $k=0,1,...,n$ & $np$ & $np(1-p)$ \\ 
		\hline negatív binomiális \newline $Negbin(n,p)$ & Hányadikra következik be $n.$ alkalommal egy $p$ valószínűségű esemény. & $P(X=k) = {k-1 \choose n-1}p^{n}(1-p)^{k-n}$ \newline $k=n,n+1,...$ & $\frac{n}{p}$ & $\frac{n(1-p)}{p^{2}}$ \\ 
		\hline Poisson \newline $Poi(\lambda)$ & Ritka esemény. & $P(X=k) = \frac{\lambda^k}{k!}e^{-\lambda}$ & $\lambda$ & $\lambda$ \\ 
		\hline 
	\end{tabular} 
	
	\subsection{Folytonos valószínűségi változók}
	
	Egy $\xi$ valószínűségi változó abszolút folytonos, ha létezik olyan $f(x)$ függvény, amelyre $F(x) = \int_{-\infty}^{x}{f(t) dt}$. Ilyenkor $f(x)$ \textit{sűrűségfüggvény}. ($F(x)$ pedig az eloszlásfüggvény.) \\
	Másik megfogalmazás: $\forall a < b$-re $P(a < \xi < b) = \int_{a}^{b}{f(t)dt}$, $F_{\xi}(x) = P(\xi < x) = \lim\limits_{a \to -\infty}{P(a < \xi \ b)} = \int_{-\infty}^{x}{f(t)dt}$. \\
	Sűrűségfüggvény tulajdonságai:
	\begin{enumerate}
		\item $f(x) = F'(x)$
		\item $f(x) \geq 0$
		\item $\int_{-\infty}^{\infty}{f(x)dx} = 1$
	\end{enumerate}
	
	\begin{tabular}{|p{3cm}|c|c|c|c|}
		\hline \textbf{Név} & \textbf{Eloszlásfüggvény} & \textbf{Sűrűségfüggvény} & \textbf{$EX$} & \textbf{$D^{2}X$} \\ 
		\hline egyenletes \newline $E(a,b)$ 
		& $\left\{\begin{array} {lr}
					0 & x \leq a \\
					\frac{x-a}{b-a} &  a < x \leq b \\
					1 & b < x 
			\end{array}\right.$
		& $\left\{\begin{array} {lr}
					\frac{1}{b-a} & a < x \leq b \\
					0 &  otherwise
			\end{array}\right.$
		& $\frac{a+b}{2}$
		& $\frac{(b-a)^2}{12}$ \\
		\hline exponenciális \newline $Exp(\lambda)$ 
		&  $\left\{\begin{array} {lr}
					1 - e^{-\lambda x} & x \geq 0 \\
					0 &  otherwise
			\end{array}\right.$
		&  $\left\{\begin{array} {lr}
					\lambda \cdot e^{-\lambda x} & x \geq 0 \\
					0 &  otherwise
			\end{array}\right.$
		& $\frac{1}{\lambda}$
		& $\frac{1}{\lambda^{2}}$ \\
		\hline normális \newline $N(m,\sigma^2)$ & $...$ & $\frac{1}{\sqrt{2 \pi}\sigma}e^{-\frac{(x-m)^2}{2\sigma^2}}$ $x \in \mathbb{R}$ & $m$ & $\sigma^2$ \\
		\hline standard normális \newline $N(0,1^2)$ & $\Phi(x)=...$ & $\frac{1}{\sqrt{2 \pi}}e^{-\frac{x^2}{2}}$ $x \in \mathbb{R}$ & $0$ & $1$ \\
		\hline gamma \newline $\Gamma(\alpha,\lambda)$ 
		& $...$ 
		& $\left\{\begin{array} {lr}
					\frac{1}{\Gamma(\alpha)}\lambda^{\alpha}x^{\alpha-1}e^{-\lambda x} & x \geq 0 \\
					0 &  otherwise
			\end{array}\right.$ 
		& $\frac{\alpha}{\lambda}$
		& $\frac{\alpha}{\lambda^2}$ \\
		\hline 
	\end{tabular}
	
	\subsection{Fogalmak}
	
	\begin{itemize}
		\item \textit{Konvolúció}: $X,Y$ független valószínűségi változók, konvolúciójuk az $X+Y$ v. v.
		\item \textit{Függetlenség}: $P(\xi_1<x_1, ..., \xi_n<x_n) = \prod_{i=1}^{n}{P(\xi_i<x_i)}$ vagy diszkrét esetben: $P(\xi_1)x_1, ..., \xi_n=x_n) = \prod_{i=1}^{n}{P(\xi_i=x_i)}$
		\item \textit{Várható érték}: $(\Omega, \mathcal{A}, P)$ valószínűségi mező, $X:\Omega \to \mathbb{R}$ valószínűségi változó, $EX = \int_{\Omega}{XdP}$, ha ez létezik. Diszkrét esetben $EX = \sum_{k}{x_k \cdot p_k}$, ha abszolút konvergens. Abszolút folytonos esetben $EX = \int_{-\infty}^{\infty}{x \cdot f(x)dx}$, ha abszolút folytonos.
		\item \textit{Szórásnégyzet}: $D^{2}X = E((X-EX)^2) = EX^2-E^{2}X$
		\item \textit{$l.$ momentum}: $EX^l = \int_{\Omega}{x^{l}dP}$, ha létezik.
		\item \textit{Szórás}: $DX = \sqrt{D^{2}X}$
		\item \textit{Kovariancia}: $cov(X,Y) = E((X-EX)(Y-EY) = E(XY) - EX \cdot EY$. Ha $cov(X,Y) = 0$, akkor $X$ és $Y$ korrelálatlan. (Megjegyzés: ha két v.v. független, akkor $cov(X,Y) = 0$, vagyis korrelálatlanok; illetve $cov(X,X) = D^{2}X$.)
		\item \textit{Korreláció}: $R(X,Y) = \frac{cov(X,Y)}{DX \cdot DY}$, két v.v. lineáris kapcsolatát méri. $R>0 \to$ pozitív, $R<0 \to$ negatív; $R^2 \sim 1 \to$ erős, $R^2 \sim 0.5 \to$ közepes, $R^2 \sim 0 \to$ gyenge.
	\end{itemize}
	
	\section{Nagy számok törvénye}
	
	\subsection{Gyenge törvény}
	
	$X_1, X_2, ...$ függetlenek, azonos eloszlásúak, $EX_i = m < \infty$, $D^{2}X_i = \sigma^2 < \infty$. \\
	$P(\frac{X_1 + ... + X_n}{n} - m \geq \varepsilon) \rightarrow 0$ $(n \to \infty)$ $\forall \varepsilon > 0$-ra (sztochasztikus konvergencia).
	
	\subsection{Erős törvény}
	
	$X_1, X_2, ...$ függetlenek, azonos eloszlásúak, $EX_1 = m < \infty$, $D^{2}X_1 = \sigma^2 < \infty$. \\
	$\frac{X_1 + ... + X_n}{n} \rightarrow m$ $(n \to \infty)$ 1 valószínűséggel. \\
	Megjegyzés: Csebisev-egyenlőtlenséggel bizonyítjuk. ($\frac{\sigma^2}{n \varepsilon^2} \to 0$ $(n \to \infty)$)
	
	\subsubsection{Csebisev-egyenlőtlenség}
	
	$EX$ véges. \\
	Ekkor $P(|X-EX| \geq \lambda) \leq \frac{D^{2}X}{\lambda^2}$ \\
	Megjegyzés: Bizonyítás Markov-egyenlőtlenséggel.
	
	\subsubsection{Markov-egyenlőtlenség}
	
	$X \geq 0, c > 0$. \\
	Ekkor $P(X \geq c) \leq \frac{EX}{c}$
	
	\subsection{Konvergenciafajták}
	
	$\xi_n \to \xi$, vagyis $\xi$ konvergens.
	\begin{itemize}
		\item \textit{sztochasztikusan}: ha $\forall \varepsilon > 0$-ra $P(|\xi_n - \xi| \geq \varepsilon) \rightarrow 0$ $(n \to \infty)$.
		\item \textit{1 valószínűséggel (majdnem mindenütt)}: ha $P(\omega : \xi_n(\omega) \to \xi(\omega)) = 1$.
		\item \textit{$L^p$-ben}: ha $E(|\xi_n - \xi|^p) \rightarrow 0$ $(n \to \infty)$ ($p>0$ rögzített).
		\item \textit{eloszlásban}: ha $F_{\xi_n}(x) \rightarrow F_{\xi}(x)$ $(n \to \infty)$ az utóbbi minden folytonossági pontjában.
	\end{itemize}
	\textit{Kapcsolataik}: 1 valószínűségű és $L^p$-beli a legerősebb, ezekből következik a sztochasztikus, ebből pedig az eloszlásbeli.
	
	\section{Centrális határeloszlás tétel}
	
	$X_1, X_2, ...$ függetlenek, azonos eloszlásúak, $EX_1 = m < \infty$, $D^{2}X_1 = \sigma^2 < \infty$. \\
	Ekkor $\frac{X_1 + ... + X_n - nm}{\sqrt{n}\sigma} \rightarrow N(0,1)$ $(n \to \infty)$ eloszlásban, azaz $P(\frac{X_1 + ... + X_n - nm}{\sqrt{n}\sigma} < x) \rightarrow \Phi(x)$ $(n \to \infty)$.
	
	\section{Statisztikai mező}
	
	$(\Omega, \mathcal{A}, \mathcal{P})$ hármas, ha $\mathcal{P} = \{P_{\vartheta}\}_{\vartheta \in \Theta}$ és $(\Omega, \mathcal{A}, P_{\vartheta})$ Kolmogorov-féle valószínűségi mező $\forall \vartheta \in \Theta$-ra.
	
	\subsection{Fogalmak}
	
	\begin{itemize}
		\item \textit{Minta}: $\underline{\xi} = (\xi_1,...,\xi_n): \Omega \to \Xi \in \mathbb{R}^n$. ($\xi_i$ valószínűségi változó)
		\item \textit{Mintatér}: $\Xi$, minta lehetséges értékeinek halmaza, gyakran $\mathbb{R}^n, \mathbb{Z}^n$.
		\item \textit{Minta [realizációja]}: $\underline{x} = (x_1,...,x_n)$, konkrét megfigyelés.
		\item \textit{Statisztika}: $T: \Xi \to \mathbb{R}^k$.
		\item \textit{Statisztika alaptétele}: (Glivenko--Cantelli-tétel) $\xi_1, \xi_2, ...$ független, azonos eloszlású $F$ eloszlásfüggvénnyel. Ekkor az $F_n$ tapasztalati eloszlásfüggvényre teljesül, hogy $\sup_{-\infty<x<\infty}{|F_n(x) - F(x)|} \to 0$ ($n \to \infty$) 1 valószínűséggel.
	\end{itemize}
	
	\section{Statisztikai becslések}
	
	$(\Omega, \mathcal{A}, \mathcal{P})$ statisztikai mező, $\vartheta \in \Theta$, $P_{\vartheta}(\xi_1<x_1,...,\xi_n<x_n) = F_{\vartheta}(\underline{x})$ \\
	$T(\underline{\xi})$ a $\vartheta$ \textit{becslése}, ha $T: \mathbb{R}^n \to \Theta$. \\
	$T(\underline{\xi})$ a $h(\vartheta)$ \textit{becslése}, ha $T: \mathbb{R}^n \to h(\Theta)$. \\
	\textit{Torzítatlanság}: $T(\underline{\xi})$ torzítatlan becslése $h(\vartheta)$-nak, ha $E_{\vartheta}T(\underline{\xi}) = h(\vartheta)$ $\forall \vartheta \in \Theta$. \\
	\textit{Aszimptotikusan torzítatlan}: $T(\underline{\xi})$ aszimptotikusan torzítatlan a $h(\vartheta)$-ra, ha $E_{\vartheta}T(\underline{\xi}) \to h(\vartheta)$ $(n \to \infty)$ $\forall \vartheta \in \Theta$. \\
	A $T_1$ torzítatlan becslés \textit{hatásosabb} $T_2$ torzítatlan becslésnél, ha $D_{\vartheta}^{2}T_1 \leq D_{\vartheta}^{2}T_2$ $\forall \vartheta \in \Theta$. \\
	Hatásos, ha minden más torzítatlan becslésnél hatásosabb. Ha van hatásos becslés, akkor az egyértelmű.
	
	\begin{itemize}
		\item \textit{Maximum-likelihood becslés}: Likelihood függvény: $L(\vartheta, \underline{x}) = \left\{\begin{array} {lr}
		P_{\vartheta}(\underline{\xi} = \underline{x}) & diszkr. \\
		f_{\vartheta, \underline{\xi}}(\underline{x}) & absz. folyt.
		\end{array}\right.$ \\
		Független esetben: $L(\vartheta, \underline{x}) = \left\{\begin{array} {lr}
		\prod_{i=1}^{n}{P_{\vartheta}(\xi_i = x_i)} & diszkr. \\
		\prod_{i=1}^{n}{f_{\vartheta, \xi_i}(x_i)} & absz. folyt.
		\end{array}\right.$\\
		$\hat{\vartheta}$ a $\vartheta$ ismeretlen paraméter maximum-likelihood becslése, ha $L(\hat{\vartheta},\underline{\xi}) = \max_{\vartheta \in \Theta}{L(\vartheta,\underline{\xi})}$.
		\item \textit{Momentum-módszer becslés}: $\vartheta = (\vartheta_1, ..., \vartheta_k)$, $\xi_1, ..., \xi_n$, $l.$ momentum: $M_l(\underline{\vartheta}) = E_{\underline{\vartheta}}\xi_{i}^{l}$, tapasztalati $l.$ momentum: $\hat{M}_l = \frac{\sum_{i=1}^{n}{\xi_i^l}}{n}$. \\ $\hat{\underline{\vartheta}}$ a $\underline{\vartheta}$ momentum módszer szerinti becslése, ha megoldása az $M_l(\underline{\vartheta}) = \hat{M}_l$, $l=1..k$ egyenletrendszernek.
	\end{itemize}
	
	\section{Hipotézisvizsgálat}
	
	Felteszünk egy hipotézist, és vizsgáljuk, hogy igaz-e. Elfogadjuk vagy elutasítjuk. Lehet paraméteres vagy nem paraméteres, vizsgálhatjuk várható értékek, szórások egyezőségét, értékét, teljes eseményrendszerek függetlenségét. Illeszkedésvizsgálattal megállapíthatjuk, hogy a valószínűségi változók adott eloszlásfüggvényűek-e, homogenitásvizsgálattal pedig azt, hogy ugyanolyan eloszlású-e két minta. \\
	$H_0$: nullhipotézis, $\vartheta \in \Theta_0$; $H_1$: ellenhipotézis, $\vartheta \in \Theta_1$; $\Theta = \Theta_0 \bigcup \Theta_1$. \\
	\textit{Egy- és kétoldali vizsgálat}: Kétoldali ellenhipotézisnél a nem egyezőséget tesszük fel, egyoldalinál valamilyen relációt. Kétoldalinál a próba értékének abszolút értékét vizsgáljuk, hogy az elfogadási tartományon belül van-e, ekkor például az u-próbánál az adott hibaszázalékot meg kell felezni a számításhoz, hiszen a $\Phi$ függvény szimmetrikus az $y$ tengelyre.
	\begin{itemize}
	\item \textit{Statisztikai próba}: $\Xi = \Xi_e \bigcup \Xi_k$ (diszjunkt halmazok) elfogadási és kritikus tartomány. Ez a felbontás a statisztikai próba. Ha a megfigyelés eleme a kritikus tartománynak, akkor elutasítjuk a nullhipotézist, ha nem eleme, akkor elfogadjuk. $T(\underline{x}) = \left\{\begin{array} {lr} 1 & x \in \Xi_k \\ 0 & otherwise \end{array}\right.$
	\item \textit{Elsőfajú hiba}: $H_0$ igaz, de elutasítjuk. Valószínűsége: $P_{\vartheta}(\underline{\xi} \in \Xi_k)$, $\vartheta \in \Theta_0$.
	\item \textit{Másodfajú hiba}: $H_0$ hamis, de elfogadjuk. Valószínűsége: $P_{\vartheta}(\underline{\xi} \notin \Xi_k)$, $\vartheta \in \Theta_1$. \\
	Az a cél, hogy ezek a hibák minél kisebbek legyenek. Egymás kárára javítható a két valószínűség, ha a megfigyelések száma rögzített.
	\item \textit{Próba terjedelme}: $\alpha$ a próba terjedelme, ha $P_{\vartheta}(\underline{\xi} \in \Xi_k) \leq \alpha$, $\vartheta \in \Theta_0$. \\
	$\alpha$ a próba pontos terjedelme, ha $\sup_{\vartheta \in \Theta_0}{P_{\vartheta}(\underline{\xi} \in \Xi_k) = \alpha}$.
	\end{itemize}
	
	\section{Klasszikus statisztikai próbák}
	
	\begin{itemize}
		\item \textit{u-próba}: Feltételezzük, hogy a minta normális eloszlású ($\xi_i \sim N(m, \sigma^2)$), $i=1..n$, és hogy a szórás ismert.
			\begin{itemize}
				\item \textit{Egymintás}: A nullhipotézis az, hogy a várható érték megegyezik-e egy konkrét értékkel ($m_0$), másképpen fogalmazva azt vizsgáljuk, hogy a mintabeli átlag nem tér-e el szignifikánsan $m_0$-tól. Tehát $H_0: m = m_0$, és kétoldali esetben $H_1: m \neq m_0$, egyoldaliban pedig például $H_1: m \geq m_0$ vagy $H_1: m < m_0$. \\
				Az u-próba értéke: $u = \sqrt{n}\frac{\overline{\xi} - m_0}{\sigma}$. Ha igaz a nullhipotézis, akkor ez közel standard normális eloszlású. \\
				$\varepsilon$ hibavalószínűséggel vizsgáljuk a hipotézist, ehhez szükségünk van a $\Phi(u_{1-\varepsilon}) = 1 - \varepsilon$ értékre. \\
				Kétoldali esetben $H_0$-t elutasítjuk, ha $|u| > u_{1-\frac{\varepsilon}{2}}$, és elfogadjuk, ha $|u| \leq u_{1-\frac{\varepsilon}{2}}$. \\
				Egyoldali esetben $u > u_{1-\varepsilon}$ (jobb) és $u < u_{1-\varepsilon}$ (bal) esetét vizsgáljuk, ezen esetekben utasítjuk el $H_0$-t.
				\item \textit{Kétmintás}: Itt a feltételek a következők: $\xi_i \sim N(m_1, \sigma_1^2)$, $i=1..n$ és $\eta_j \sim N(m_2, \sigma_2^2)$, $j=1..m$. A szórások szintén ismertek. $H_0: m_1 = m_2$, és $u = \frac{\overline{\xi} - \overline{\eta}}{\sqrt{\frac{\sigma_1^2}{n} + \frac{\sigma_2^2}{m}}}$. $H_1 : m_1 > m_2$, ez a felső (jobb?) oldali, $H_1 : m_1 < m_2$ pedig az alsó (bal?) ellenhipotézis.
			\end{itemize}
		\item \textit{t-próba}:	Ennél a próbánál nem ismert a szórás, viszont ugyanúgy normális eloszlást feltételezünk, mint az u-próbánál. $\xi_i \sim N(m, \sigma^2)$, $i=1..n$.
			\begin{itemize}
				\item \textit{Egymintás}: $H_0: m = m_0$. Ellenhipotézis az u-próbához hasonlóan. $t = \sqrt{n}\frac{\overline{\xi} - m_0}{\sqrt{\sigma_{*}^2}}$, ahol $\sigma_{*}^2$ a korrigált tapasztalati szórásnégyzet, amit a mintából számíthatunk ki. (Megjegyzés: $n$ helyett $n-1$-gyel osztunk a képletben.) Ez az érték $t$-eloszlású $H_0$ esetén, ami $n-1$ szabadságfokú. Más néven szokás ezt a próbát Student-próbának is nevezni. 
				\item \textit{Kétmintás}: $\xi_i \sim N(m_1, \sigma_1^2)$, $i=1..n$ és $\eta_j \sim N(m_2, \sigma_2^2)$, $j=1..m$. Ez esetben sem ismert a szórás, viszont feltételezzük, hogy a két minta szórása megegyezik. Ekkor $t_{n+m-2} = \sqrt{\frac{nm(n+m-2)}{n+m}}\frac{\overline{\xi} - \overline{\eta}}{\sqrt{\sum{(\xi_i - \overline{\xi})^2} + \sum{(\eta_j - \overline{\eta})^2}}}$. $n+m-2$ a próba szabadságfoka.
			\end{itemize}
		\item \textit{f-próba}: Két minta esetén használható. Ez a próba szórások egyezőségének vizsgálatára alkalmas, tehát itt $H_0 : \sigma_1 = \sigma_2$. Ha a két minta szórásnégyzete megegyezik, akkor a hányadosuk $1$-hez tart. $f_{n-1,m-1} = max(\frac{\sigma_1^2}{\sigma_2^2}, \frac{\sigma_2^2}{\sigma_1^2})$. A két szabadsági fok közül az első az $f$ számlálójához tartozó minta elemszáma $- 1$, a második a nevezőjéhez.
		\item \textit{Welch-próba}: Más néven d-próba. Hasonló, mint a kétmintás t-próba, de itt a szórások egyezőségét nem kell feltenni. Szabadsági foka bonyolult képlettel számítható.
		\item \textit{szekvenciális próbák}: $V_n = \frac{\prod{f_1(x_i)}}{\prod{f_0(x_i)}} = \frac{L_1(\underline{x})}{L_0(\underline{x})}$. $f_0$ a nullhipotézis szerinti sűrűségfüggvény, $f_1$ az ellenhipotézis szerinti. Adott egy $A$ és egy $B$ érték, $A<B$. Ha $V_n \geq B$, akkor elutasítjuk $H_0$-t, ha $V_n \leq A$, akkor elfogadjuk, és ha $A < V_n < B$, akkor új mintaelemet veszünk. \\
		\textit{Stein tétele} szerint $N$ 1 valószínűséggel véges. $N = min\{n: V_n \leq A \vee V_n \geq B \}$.
		\item \textit{Minőség-ellenőrzés}: $n_1$ elemet nézünk, $c_1 < c_2$ és $c_3$ határértékek. Ha $X_1 \leq c_1$, akkor elfogadjuk $H_0$-t, ha $X_1 \geq c_2$, akkor elutasítjuk. Ha $c_1 < X_1 < c_2$, akkor megnézünk $n_2$ elemet, és ha $X_1 + X_2 \leq c_3$, akkor szintén elfogadjuk $H_0$-t. A várható mintaelemszám méri a hatékonyságát.
		\item \textit{$\chi^2$-próba}: $H_0: A_1,...,A_n$ teljes eseményrendszer. $P(A_i) = p_i, i=1..n$, $\nu_i$ a gyakoriság. Ha teljesül a nullhipotézis, akkor $\frac{\nu_i}{n} \sim p_i$. \\
		$\chi^2 = \sum{\frac{(\nu_i - np_i)^2}{np_i}}$. Ez $\chi^2$ eloszlású, aminek $r-1$ szabadságfoka van. $r$ az összeadott csoportok száma. (Megjegyzés: ha túl kicsi lenne 1-1 csoportban a gyakoriság, akkor azokat összevonjuk.) \\
		$\chi^2$-próbát használhatunk illeszkedés-, homogenitás- és függetlenségvizsgálatra is. (Megjegyzés: más képlet van mindhez.) 
		\item \textit{Egyéb próbák}:
			\begin{itemize}
				\item \textit{Kolmogorov--Szmirnov-próba}: 2 tapasztalati eloszlásfüggvény megegyezik-e (homogenitásvizsgálat), vagy 1 minta esetén megegyezik-e valamilyen eloszlásfüggvénnyel. $D_{m,n} = \max_{x}{|F_n(x) - G_m(x)|}$. $X_i$ $F$ eloszlásfüggvénnyel, $Y_j$ $G$-vel. $H_0 : F \equiv G$.
				\item \textit{Előjel-próba}: Hányszor teljesül, hogy valami pozitív.
				\item \textit{Wilcoxon-próba}: (rangstatisztika), $P(X > Y) = \frac{1}{2}$ tesztelésére összeszámoljuk, hogy hány párra teljesül, hogy $X_i > Y_j$. 
			\end{itemize}
	\end{itemize}
	
\end{document}