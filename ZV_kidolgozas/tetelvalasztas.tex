\documentclass[a4paper,12pt]{article}
\usepackage[utf8]{inputenc}
\begin{document}
	\section*{Tételek felosztása}
	\paragraph{1. csoport:}
	1-2-3 (anal1 - anal többi - nummód)
	\begin{description}
		\item[1:] Dóri
		\item[2:] Andris
		\item[3:] Ádám
	\end{description}
	\paragraph{2. csoport:}
	4-5-7 (dimat - valszám - prog)
	\begin{description}
		\item[4:] Andris
		\item[5:] Dóri
		\item[7:] Ádám
	\end{description}
	
	\paragraph{3. csoport:}
	8-9-10 (szofttech - fordprog - prognyelvek)
	\begin{description}
		\item[8:] Andris
		\item[9:] Dóri
		\item[10:] Ádám
	\end{description}
	\paragraph{4. csoport:}
	15-16-17 (oprendszerek - számhálók - OR)
	\begin{description}
		\item[15:] Dóri
		\item[16:] Andris
		\item[17:] Ádám
	\end{description}
	\paragraph{5. csoport:}
	11/12-13/14-18/19 (fonya/logika - algo1/algo2 - adatb1/adatb2)
	Kettőt válassz külön csoportból, ne maradjon senkinek 2 ugyanabból a párból!
	\begin{description}
		\item[Ádám:] Logika (12.), Adatb2 (19.)
		\item[Andris:] Fonya (11.), Algo2 (14.)
		\item[Dóri:] Algo1 (13.), Adatb1 (18.)
	\end{description}
	\paragraph{6. csoport:}
	6 (MI)
	\begin{description}
		\item[6:] Dóri
	\end{description}
\end{document}
